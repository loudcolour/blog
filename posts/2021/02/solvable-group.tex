\documentclass[10pt,a4paper]{oblivoir}

\input{./preamble-en.tex}

\usepackage{fapapersize}
\usefapapersize{*,*,25mm,*,25mm,*}

\title{Solvable groups}
\author{S. Hyeon}

\begin{document}

\maketitle
\tableofcontents

\section{Definition of solvable groups}

\begin{definition}[commutator group] Let $G$ be a group.
    \begin{itemize}
        \item For $x,y\in G$, $[x,y] \triangleq xyx^{-1}y^{-1} \in G$ is called the \emph{commutator} of $x,y$.
        \item $D(G) \triangleq [G,G] = \langle [x,y] | x,y\in G \rangle$ is called the \emph{commutator group} of $G$.
    \end{itemize}
\end{definition}

\begin{remark} \hfill
    \begin{itemize}
        \item For a group $G$ and its elements $x,y\in G$, $[x,y] = 1_G$ is equivalent to $xy=yx$.
        \item For a group $G$, $G$ is abelian if and only if $D(G)$ is trivial.
    \end{itemize}
\end{remark}

\begin{example} \hfill
    \begin{enumerate}
        \item $D(S_n) = A_n$. This can be shown by the fact that for every $\sigma, \tau \in S_n$,
            $[\sigma,\tau] = \sigma\tau\sigma^{-1}\tau^{-1} \in A_n$ (thus, $D(S_n) \subset A_n$),
            and $[(ij), (ik)] = (ijk)$, while $A_n$ is generated by 3-cycles of $S_n$, which concludes that $D(S_n) \supset A_n$.
        \item For $n\geq 5$, $D(A_n) = A_n$. We can show this by a proof which is similar with the one we used right before, with the fact that
            $[(ijl), (ikm)] = (ijk)$ for $i,j,k \in X_n$ and $l,m\in X_n\setminus \left\{ i,j,k \right\}$.
    \end{enumerate}
\end{example}

Let us say $H$ is a subgroup of a group $G$, and also $N \vartriangleleft G$.
Then,
\begin{equation}
    D(HN/N) = \langle [xN,yN] | x,y\in H \rangle
            = \langle [x,y]N | x,y\in H \rangle
            = D(H)N/N
\end{equation}
Especially in the case of $G = H$, $D(G/N) = D(G)N/N$.

\begin{proposition} Let $G$ be a group.
    \label{PROP: 1}
    \begin{enumerate}
        \item For $H$, which is a subgroup of $G$, $D(H) \subset D(G)$.
        \item $D(G) \vartriangleleft G$.
    \end{enumerate}
\end{proposition}

\begin{proof}
    Statement 1 is trivial. For statement 2, we can easily check that
    $gD(G)g^{-1} \subset D(G)$ for every $g\in G$.
\end{proof}

\begin{proposition} Let $G$ be a group and $N \vartriangleleft G$. Then the followings are equivalent.
    \label{PROP: 2}
    \begin{enumerate}
        \item $G/N$ is abelian.
        \item $D(G)\subset N$.
    \end{enumerate}
    By this, $G/D(G)$ is abelian.
\end{proposition}

\begin{proof}
    $G/N$ is abelian. $\iff$ $D(G/N)$ ($= D(G)N/N$) is trivial. $\iff$
    $D(G) \subset N$.
\end{proof}

\begin{remark}
    Let us say $G'$ is an abelian group and $f: G\to G'$ is a homomorphism.
    This implies that there exists the unique homomorphism $g: G/D(G) \to G'$,
    which is derived by $f$, because $G/\text{Ker}(f) \cong \text{Im}(f) \subset G'$ is abelian and
    thus $D(G) \subset \text{Ker}(f)$ by Proposition~\ref{PROP: 2}, which invokes the decomposition of homomorphism $f$.
\end{remark}

For a group $G$, we can get a sequence of group by layering commutator groups.
More specifically, Let's say $D_0(G) \triangleq G$ and $D_{i+1}(G) = D(D_i(G))$
for every $i\geq 0$, then we get a decreasing sequence $G = D_0(G) \supset D_1(G) \supset D_2(G) \supset \cdots$.

\begin{definition}[solvable group]
    $G$ is said to be \emph{solvable} if
    there exists $n\geq 0$ satisfying $D_n(G) = \left\{ 1 \right\}$.
\end{definition}

\begin{example}
    Every abelian group is solvable.
\end{example}

\begin{lemma}
    \label{LEM: 1}
    Let us say $G$ is a solvable group.
    \begin{enumerate}
        \item Let $H$ be a subgroup of $G$, then $H$ is also solvable.
        \item For $N \vartriangleleft G$, $G/N$ is solvable.
    \end{enumerate}
\end{lemma}

\begin{proof}
    Statement 1 is implied by $\forall i\geq 0, D_i(H) \subset D_i(G)$, which
    can be shown by induction. Statement 2 is also implied by $\forall i\geq 0,
    D_i(G/N) = D_i(G)N/N$, which also can be shown by induction, more specifically,
    $D_{i+1}(G/N) = D(D_{i}(G/N)) = D(D_{i}(G)N/N) = D_{i+1}(G)N/N$ for every $i\geq 0$
    provided that $D_{i}(G/N) = D_{i}(G)N/N$,
    and $D_0(G/N) = G/N = D_0(G)N/N$ for the first place.
\end{proof}

\begin{theorem}
    \label{THM: 1}
    Let $G$ be a group and $N \vartriangleleft G$. Then the followings are equivalent.
    \begin{enumerate}
        \item $G$ is solvable.
        \item $G/N$ and $N$ is solvable at the same time.
        \item There exists a chain $G = G_0 \vartriangleright G_1 \vartriangleright \cdots \vartriangleright G_n = \left\{ 1 \right\}$,
            where $G_i/G_{i+1}$ is abelian for every $i\geq 0$.
    \end{enumerate}
\end{theorem}

\begin{proof} \hfill
    \begin{itemize}
        \item Statement 1 implies statement 2. This is said by Lemma~\ref{LEM: 1}.
        \item Statement 2 implies statement 1. Let's say $D_n(N) = \left\{ 1 \right\}$ and $D_m(G/N) = \left\{ 1 \right\}$.
            Then, $ \left\{ 1 \right\} = D_m(G/N) = D_m(G)N/N$, $D_m(G) \subset N$. Thus
            $D_{n+m}(G) \subset D_n(N) = \left\{ 1 \right\}$.
        \item Statement 1 also implies statement 3 by putting $G_i \triangleq D_i(G)$, and Proposition~\ref{PROP: 1}, Proposition~\ref{PROP: 2}.
        \item Statement 3 implies statement 1. With induction on $n$ starting from $n=0$,
            $G = G_0 = \left\{ 1 \right\}$ is solvable, and also
            $G = G_0$ is solvable if $G_1$ and $G/G_1$ is solvable at the same time, which we can acquire from
            the supposition of the induction and the fact $G_i/G_{i+1}$ is abelian respectively.
    \end{itemize}
\end{proof}

\begin{example} \hfill
    \begin{itemize}
        \item A group $G$ is said to be \emph{simple} if
            it is not abelian and has trivial subgroup as its only normal subgroup.
            Every simple group is \emph{not} solvable.
        \item $S_4$ and $A_4$ are both solvable because there exists a chain fulfilling
            conditions from Theorem~\ref{THM: 1}: $S_4 \vartriangleright A_4 \vartriangleright V_4 \vartriangleright \left\{ 1 \right\}$.
        \item For $n\geq 5$, $S_n$ and $A_n$ is not solvable. Especially, $A_n$ is simple when $n\geq 5$.
    \end{itemize}
\end{example}

\begin{definition}[nilpotent group]
    $G$ is said to be \emph{nilpotent group} if there exists a sequence of
    group $G = G_0 \supset G_1 \supset \cdots \supset G_n = \left\{ 1 \right\}$
    which fulfills following conditions for every $i=0,\ldots,n-1$:
    \begin{enumerate}
        \item $G_{i+1} \vartriangleleft G$.
        \item $G_i/G_{i+1} \subset Z(G/G_{i+1})$.
    \end{enumerate}
\end{definition}

\begin{remark}
    For every non-trivial nilpotent group $G$, $Z(G)$ is not trivial.
\end{remark}

\begin{proposition}
    Every nilpotent group $G$ is solvable.
\end{proposition}

\begin{proof}
    By $G_{i+1} \vartriangleleft G$, $G_{i+1} \vartriangleleft G_i$ is obvious.
    Also, by $D(G_i) = [G_i, G_i] \subset [G_i, G] \subset G_{i+1}$,
    $G_i/G_{i+1}$ is abelian by Proposition~\ref{PROP: 2}.
\end{proof}

\begin{proposition}
    Every $p$-group is nilpotent group.
\end{proposition}

\begin{proof}
    Let's say $G$ is non-trivial $p$-group.
    By class equation, it is easy to prove the center of a non-trivial $p$-group
    is not trivial. We define a sequence of set $Z_i$ by $Z_0 = \left\{ 1 \right\}$ and
    $Z_{i+1} = \left\{ x\in G \mid xZ_i \in Z(G/Z_i) \right\} \supset Z_i$.
    This is possible because we can show that $Z_i$ is a normal subgroup of $G$ for every $i\in\SetN$ by induction.
    Then, we can directly show that $Z_{i+1}/Z_i = Z(G/Z_i)$.
    For $i\in\SetN$ which satisfies $Z_i \subsetneq G$, $G/Z_i$ is also $p$-group,
    thus $Z_{i+1}/Z_{i}$ is not trivial. Therefore, we can get a sequence
    $ \left\{ 1 \right\} = Z_0 \subsetneq Z_1 \subsetneq \cdots \subsetneq Z_n = G$.
\end{proof}

\begin{example}
    Let $G$ be a group and its order be $56$. Then, $G$ is solvable.
    $56 = 2^3\cdot 7$, and there exists two possibilities on the number of
    7-Sylow subgroup: 1 and 8. If there exists only one 7-Sylow subgroup $P_7$,
    i.e. $P_7 \vartriangleleft G$, $P_7$ and $G/P_7$ is 7-group and 2-group respectively,
    thus $G$ is solvable by Lemma~\ref{LEM: 1}.

    If there exist 8 7-Sylow subgroups of $G$, there are 48 elements of $G$
    whose order is 7. (Let us say $P_7$ and $P_7'$ are not identical and both
    7-Sylow subgroups.  If $x\in P_7 \cap P_7'$ and $x \neq 1$, $P_7 = \langle
    x \rangle = P_7'$.  By contradiction, $P_7 \cap P_7'$ is trivial.) Thus
    there exist at most 8 elements whose order is 1, 2, 4, or 8 on $G$.
    This implies there is only one 2-Sylow subgroup $P_2$ i.e. $P_2 \vartriangleleft G$,
    and thus $G$ is solvable as same as the case of $P_7 \vartriangleleft G$.
\end{example}

\end{document}

